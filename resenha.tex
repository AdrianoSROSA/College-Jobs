\documentclass[article]{abntex2}
\usepackage{graphicx}

\setlrmarginsandblock{3cm}{3cm}{*}
\setulmarginsandblock{3cm}{3cm}{*}
\checkandfixthelayout

%-------------------title page-----------------------
\begin{document}
\begin{titlepage}
  \centering
  \includegraphics[width=0.3\textwidth]{ifsp.jpg}\par\vspace{1cm}
   {\scshape\LARGE Análise e Desenvolvimento de Sistemas \par}
   \vspace{1.5cm}
%   {\scshape\Large Resenha \par}
   \vspace{1.5cm}
   {\huge\bfseries Sistemas operacionais modernos \par}
   \vspace{1cm}
   {\Large\itshape Andrew S. Tanenbaum \par}
   \vspace{1cm}
   \vfill
	 Adriano da Silva Rosa - BP300838X\par
	 Matheus Felipe Zacarias - BP3008461\par
	 Pedro Guilherme Calasans de Souza - BP3008517\par
   \vfill

	 {\large 6 de março de 2020\par}
 
\end{titlepage}

\newpage
%------------------------------text--------------------------
{\scshape\Large Resenha \par}
\vspace{1cm}
Uma das principais tarefas dos sistemas operacionais é esconder o hardware e em vez disso apresentar programas com abstrações de qualidade,limpas, elegantes e consistentes com as quais trabalhar. O Sistema Operacional é um software que opera em modo nucleo.

A arquitetura dos computadores em nivel de linguagem de maquina é primitiva e complicada de programar.
Por essa razão, todos os sistemas operacionais fornecem níveis de abstração.
Boas abstrações transformam uma tarefa impossível em duas tarefas gerenciaveis.
A primeira é definir e implementar as abstrações, e a segunda é utiliza-las para solucionar o problema à mão.
A função dos S.O. é criar boas abstrações e então implementar e gerenciar os objetos abstratos criados desse modo. 
Alem de fornecer uma alocação ordenada e controlada dos processadores, memórias e dispositivos de E/S entre os vários programas competindo por eles. 
A principal função é manter um controle sobre quais programas estão usando qual recurso, conceder recursos requisitados, contabilizar o seu uso, assim como mediar requisições conflitantes de diferentes programas e usuários.
O gerenciamento de recursos inclui a multiplexação de recursos de duas maneiras diferentes. 
No tempo diferentes programas ou usuários se revezam usando-o; e no espaço em vez de os clientes se revezarem, cada um tem direito a uma parte do recurso.

No inicio, um único grupo de pessoas projetava, construía, programava, operava e mantinha cada máquina.
Toda a programação era feita em código de máquina, ligando circuitos elétricos através de conexão de milhares de cabos a painés de ligações para controlar as funções basicas da máquina.
Com o tempo os computadores tornaram-se menores, confiáveis,manufaturados e houve a separação entre projetistas, construtores, operadores, programadores e pessoal da manutenção.

A multiprogramação divide a memória em várias partes, com uma tarefa diferente em cada partição.
Enquanto uma tarefa ficava esperando pelo término da E/S, outra podia usar a CPU.
Se um numero suficiente de tarefas pudesse ser armazenado na memória principal ao mesmo tempo, a CPU podia se manter ocupada quase 100\% do tempo.
Spooling é sempre que uma tarefa sendo executada terminava, o sistema operacional podia carregar uma nova tarefa do disco para a partição agora vazia e executá-la.
Timesharing é uma variante da multiprogramação na qual cada usuário tem um terminal on-line.
O CTSS foi o primeiro sistema de compartilhamento de tempo para fins diversos, desenvolvido no M.I.T. em um 7094 modificado.
O MULTICS é um computador utilitário que daria suporte a algumas centenas de usuários simultâneos com compartilhamento de tempo.
O conceito de cloud computting é de que computadores relativamente pequenos estejam conectados a servidores em vastos e distantes centros de processamento de dados onde toda a computação é feita com o computador local apenas executando a interface com o usuário.

O conjunto de chamadas de sistema é o coração dos sistemas operacionais e o principal diferencial entre eles.
Essas chamadas dizem o que o sitema operacional realmente faz.
Sistemas operacionais são antigos, seu objetivo inicial era substituir operadores mas hoje já esta além da multiprogramação moderna.
Teve o nascimento do UNIX, que se tornou popular no mundo acadêmico, em agências do governo e em muitas empresas.
Que deu origem a outros sistemas operacionas como o FreeBSD e Linux.
No UNIX as chamadas de sistema são divididos em quatro grupos: O primeiro grupo diz respeito à criação e ao término de processos.O segundo é para leitura e escrita de arquivos. O terceiro é para o gerenciamento de diretórios.
E o quarto grupo contém chamadas diversas.
Podendo ser estruturado de diversas maneiras os S.O. mais comuns são a hierarquia de camadas, o exonúcleo, o micronúcleo, o cliente-servidor, a máquina virtual e o sistema monolítico.\\
%-------------------------end text--------------------------
\\
{\scshape\Large Atividades \par}

\section{Quais as principais funções de um Sistema Operacional ?}
Gerenciar recursos: O S.O. deve gerenciar a utilização dos recursos fornecidos pelo hardware, como processadores, memória, dispositivos de E/S, de modo que mantenha o controle sobre qual usuário/programa utiliza qual recurso, compartilhando os recursos entre os usuários/programas de modo seguro e sem conflitos.\\
Estender a máquina: O S.O. deve oferecer ao usuário uma maneira mais acessivel de programar/utilizar o hardware do que as proprias instruções que este oferece. por exemplo: o usuário não precisa saber qual a trilha e o setor do disco se deseja gravar alguma informação, apenas faz a chamada ao sistema que estende as instruções de E/S., disponibilizando instruções mais amigáveis para estas e outras tarefas. Ou seja, o S.O. atua como uma interface entre o hardware e o ambiente de software.


\section{Qual é a diferença entre sistemas de compartilhamento de tempo e de multiprogramação?}
Compartilhamento de tempo: Possibilita que vários programas sejam executados atraves de uma divisão de tempo do processador em intervalos curtos.\\
Multiprogramação ou multitarefa: Possibilitam que enquanto um programa esteja em operação de leitura outros possam ser executados.


\section{A idéia de familia de computadores foi introduzida na década de 1960 com os computadores de grande porte System/360 da IBM. Essa idéia está ultrapassada ou ainda é valida? }
Essa idéia continua válida, uma vez que fornece, por utilizar o mesmo conjunto de instruções para uma série de sistemas, algumas vantagens como:
Portabilidade para diversos softwares, inclusive antigos; e uma maior variedade de opções para compra, que permite melhores escolhas e escalabilidade dentro de um sistema.


\section{A chamada count = write(fd,buffer,nbytes); pode retornar qualquer valor count fora nbyter? Se a resposta for sim, por quê?}
Sim, essa chamada pode retornar algum valor diferente de nbyter.\\
Principalmente por dois motivos, se um final de arquivo for encontrado antes, ou o arquivo não pôde ser lido. se a chamada falhar, por exemplo, porque fd está incorreto, ele pode retornar -1. Ele também pode falhar porque o disco está cheio e não é possivel gravar o número de bytes solicitados. Em uma terminação correta, sempre retorna nbytes.




\end{document}
